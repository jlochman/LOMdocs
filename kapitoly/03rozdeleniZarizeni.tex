\section{Rozdeleni Zarizeni}

Dalsi uzasnou veci jsou pripravene prikazy. Ty jsou v souboru
\textit{slovnik.tex}. Napriklad mail, vcetne odkazu pro poslani, se vlozi
naprosto jednoduse, staci: \mail.

Vkladani fyzikalnich velicin? Jednoduche jako facka 1000\mh, 20\gc.

Predpripravene, casto pouzivane slovni spojeni? \soucProstTepla strechou.

Prikazy se daji pripravit s parametry, napriklad \tempWithTolerance{20}{1}. Nebo
vami casteji pouzivany \tempRangeMax{20}{26}

A prakticky jakakoliv casteji se opakovana fraze, jednotka, tabulka se da takhle
predpripravit. Predpripravit si ceskou frazi? Pak staci zmenit soubor s ceskymi
frazemi za soubor s anglickymi a je hotovo.
